
\documentclass{article}
\title{Reading Notes for Mackey ch28 Model Comparison and Occam’s Razor}
\author{Xiang Pan}

\usepackage{url}
\usepackage{titling}
\usepackage{geometry}

% \geometry{a4paper,scale=0.9,left=10mm, right=10mm, top=3mm, bottom=20mm}
\geometry{a4paper,scale=0.9}
\usepackage{amsmath}
\usepackage{hyperref}
\usepackage{amsfonts}
\usepackage{tikz}
\def\ci{\perp\!\!\!\perp}
\usetikzlibrary{fit,positioning}

\hypersetup{
    colorlinks=true,
    linkcolor=blue,
    filecolor=blue,      
    urlcolor=blue,
    citecolor=cyan,
}


\begin{document}
\maketitle
\section{Occam’s Razor}
Occam’s razor: Accept the simplest explanation that fits the data.
Reason:
\begin{itemize}
    \item aesthetic (‘A theory with mathematical beauty is more likely to be correct than an ugly one that fits some experimental data’)
    \item past empirical success of Occam’s razor
\end{itemize}

It is essential to use proper priors – otherwise the evidences and the Occam factors are not meaningful.
$$
\frac{P\left(\mathcal{H}_{1} \mid D\right)}{P\left(\mathcal{H}_{2} \mid D\right)}=\frac{P\left(\mathcal{H}_{1}\right)}{P\left(\mathcal{H}_{2}\right)} \frac{P\left(D \mid \mathcal{H}_{1}\right)}{P\left(D \mid \mathcal{H}_{2}\right)}
$$

Occam’s Razor gives a favor of simplicity to the model.


\textbf{Model fitting}
\begin{equation}
    \begin{gathered}
    P\left(\mathbf{w} \mid D, \mathcal{H}_{i}\right)=\frac{P\left(D \mid \mathbf{w}, \mathcal{H}_{i}\right) P\left(\mathbf{w} \mid \mathcal{H}_{i}\right)}{P\left(D \mid \mathcal{H}_{i}\right)} \\
    \text { Posterior }=\frac{\text { Likelihood } \times \text { Prior }}{\text { Evidence }}
    \end{gathered}
\end{equation}


\textbf{Model Comparison}

Models $H_i$ are ranked by evaluating the evidence,
\begin{equation}
P\left(D \mid \mathcal{H}_{i}\right) \simeq \underbrace{P\left(D \mid \mathbf{w}_{\mathrm{MP}}, \mathcal{H}_{i}\right)} \times \underbrace{P\left(\mathbf{w}_{\mathrm{MP}} \mid \mathcal{H}_{i}\right) \sigma_{w \mid D}} \cdot
\end{equation}

$$
\text { Evidence } \simeq \text { Best fit likelihood } \times \text { Occam factor }
$$
$$
\text { Occam factor }=\frac{\sigma_{w \mid D}}{\sigma_{w}}
$$

Occam factor is equal to the ratio of the posterior accessible volume of $H_i$’s parameter space to the prior accessible volume.

\textbf{Occam factor for several parameters}
If the posterior is well approximated by a Gaussian, then the Occam factor is obtained from the determinant of the corresponding covariance matrix.
\begin{equation}
    P\left(D \mid \mathcal{H}_{i}\right) \simeq P\left(D \mid \mathbf{w}_{\mathrm{MP}}, H_{i}\right) \times P\left(\mathbf{w}_{\mathrm{MP}} \mid \mathcal{H}_{i}\right) \operatorname{det}^{-\frac{1}{2}}(\mathbf{A} / 2 \pi)
\end{equation}
\begin{equation}
    \mathbf{A}=-\nabla \nabla \ln P\left(\mathbf{w} \mid D, \mathcal{H}_{i}\right)
\end{equation}

\textbf{On-line learning and cross-validation.}
\begin{equation}
    \begin{aligned}
    &\log P(D \mid \mathcal{H})=\log P\left(\mathbf{t}^{(1)} \mid \mathcal{H}\right)+\log P\left(\mathbf{t}^{(2)} \mid \mathbf{t}^{(1)}, \mathcal{H}\right) \\
    &\quad+\log P\left(\mathbf{t}^{(3)} \mid \mathbf{t}^{(1)}, \mathbf{t}^{(2)}, \mathcal{H}\right)+\cdots+\log P\left(\mathbf{t}^{(N)} \mid \mathbf{t}^{(1)} \ldots \mathbf{t}^{(N-1)}, \mathcal{H}\right)
    \end{aligned}
\end{equation}

Cross-validation examines the average value of just the last term. The evidence, on the other hand, sums up how well the model predicted all the data, starting from scratch.

\bibliographystyle{plain}
% \bibliography{note4}
\appendix
\end{document}