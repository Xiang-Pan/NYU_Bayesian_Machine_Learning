
\documentclass{article}
\title{Reading Notes for ch3 Linear Models for Regression}
\author{Xiang Pan}

\usepackage{url}
\usepackage{titling}
\usepackage{geometry}

\geometry{a4paper,scale=0.8}
\usepackage{amsmath}
\usepackage{hyperref}
\usepackage{amsfonts}

\hypersetup{
    colorlinks=true,
    linkcolor=blue,
    filecolor=blue,      
    urlcolor=blue,
    citecolor=cyan,
}


\begin{document}
\maketitle
\section{Linear Basis Function Models} 
This section just reviews the basic linear regression setting. 
\section{Bias-Variance Decomposition}
$$
\text { expected loss }=(\text { bias })^{2}+\text { variance }+\text { noise }
$$
$$
\begin{aligned}
(\text { bias })^{2} &=\int\left\{\mathbb{E}_{\mathcal{D}}[y(\mathbf{x} ; \mathcal{D})]-h(\mathbf{x})\right\}^{2} p(\mathbf{x}) \mathrm{d} \mathbf{x} \\
\text { variance } &=\int \mathbb{E}_{\mathcal{D}}\left[\left\{y(\mathbf{x} ; \mathcal{D})-\mathbb{E}_{\mathcal{D}}[y(\mathbf{x} ; \mathcal{D})]\right\}^{2}\right] p(\mathbf{x}) \mathrm{d} \mathbf{x} \\
\text { noise } &=\int\{h(\mathbf{x})-t\}^{2} p(\mathbf{x}, t) \mathrm{d} \mathbf{x} \mathrm{d} t
\end{aligned}
$$

Very flexible models having low bias and high variance, and relatively rigid models having high bias and low variance.
\section{Bayesian Liear Regression}
\subsection{Error Decoposition}
$$
p(t \mid \mathbf{t}, \alpha, \beta)=\int p(t \mid \mathbf{w}, \beta) p(\mathbf{w} \mid \mathbf{t}, \alpha, \beta) \mathrm{d} \mathbf{w}
$$
$$
p(t \mid \mathbf{x}, \mathbf{t}, \alpha, \beta)=\mathcal{N}\left(t \mid \mathbf{m}_{N}^{\mathrm{T}} \boldsymbol{\phi}(\mathbf{x}), \sigma_{N}^{2}(\mathbf{x})\right)
$$
$$
\sigma_{N}^{2}(\mathbf{x})= \text{noise on the data} + \text{uncertainty associated with the parameters w.}
$$

$$
\sigma_{N}^{2}(\mathbf{x})=\frac{1}{\beta}+\phi(\mathbf{x})^{\mathrm{T}} \mathbf{S}_{N} \phi(\mathbf{x})
$$
When dataset size get unlimited, the second term goes to zero.
\subsection{Equivalent kernel(smoother matrix)}
For predictive mean,
$$
y\left(\mathbf{x}, \mathbf{m}_{N}\right)=\mathbf{m}_{N}^{\mathrm{T}} \boldsymbol{\phi}(\mathbf{x})=\beta \boldsymbol{\phi}(\mathbf{x})^{\mathrm{T}} \mathbf{S}_{N} \boldsymbol{\Phi}^{\mathrm{T}} \mathbf{t}=\sum_{n=1}^{N} \beta \boldsymbol{\phi}(\mathbf{x})^{\mathrm{T}} \mathbf{S}_{N} \boldsymbol{\phi}\left(\mathbf{x}_{n}\right) t_{n}
$$
Forming a weighted combination of the target values in which data points close to x are given higher weight than points further removed from x.
$$
y\left(\mathbf{x}, \mathbf{m}_{N}\right)=\sum_{n=1}^{N} k\left(\mathbf{x}, \mathbf{x}_{n}\right) t_{n}
$$
$$
\begin{aligned}
\operatorname{cov}\left[y(\mathbf{x}), y\left(\mathbf{x}^{\prime}\right)\right] &=\operatorname{cov}\left[\boldsymbol{\phi}(\mathbf{x})^{\mathrm{T}} \mathbf{w}, \mathbf{w}^{\mathrm{T}} \boldsymbol{\phi}\left(\mathbf{x}^{\prime}\right)\right] \\
&=\boldsymbol{\phi}(\mathbf{x})^{\mathrm{T}} \mathbf{S}_{N} \boldsymbol{\phi}\left(\mathbf{x}^{\prime}\right)=\beta^{-1} k\left(\mathbf{x}, \mathbf{x}^{\prime}\right)
\end{aligned}
$$

\section{Bayesian Model Comparison}
\subsection{posterior distribution}
$$
(\text{model posterior distribution}) \propto (\text{model prior probability distribution}) * (\text{model evidence})
$$
$$
p\left(\mathcal{M}_{i} \mid \mathcal{D}\right) \propto p\left(\mathcal{M}_{i}\right) p\left(\mathcal{D} \mid \mathcal{M}_{i}\right)
$$

prior probability distribution $p(M_i)$: allows us to express a preference for different models

model evidence $p(D|M_i)$: preference shown by the data for different models

Bayes factor $p(D | M_i)/p(D | M_j)$: the ratio of model evidences for two models.

\subsection{predictive distribution}
$$
p(t \mid \mathbf{x}, \mathcal{D})=\sum_{i=1}^{L} p\left(t \mid \mathbf{x}, \mathcal{M}_{i}, \mathcal{D}\right) p\left(\mathcal{M}_{i} \mid \mathcal{D}\right)
$$
\subsection{model evidence}
We can obtain a rough approximation to the model evidence if we assume that the posterior distribution over parameters is sharply peaked around its mode $w_{MAP}$.
$$
\ln p(\mathcal{D}) \simeq \ln p\left(\mathcal{D} \mid w_{\mathrm{MAP}}\right)+\ln \left(\frac{\Delta w_{\text {posterior }}}{\Delta w_{\text {prior }}}\right)
$$

\section{Evidence Approximation}
We set the hyperparameters to specific values determined by maximizing the marginal likelihood function obtained by first integrating over the parameters w.
$$
p(t \mid \mathbf{t}) \simeq p(t \mid \mathbf{t}, \widehat{\alpha}, \widehat{\beta})=\int p(t \mid \mathbf{w}, \widehat{\beta}) p(\mathbf{w} \mid \mathbf{t}, \widehat{\alpha}, \widehat{\beta}) \mathrm{d} \mathbf{w}
$$
update parameter iteraly.
\bibliography{ref}

\appendix
\end{document}